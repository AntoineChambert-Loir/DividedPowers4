% In this file you should put the actual content of the blueprint.
% It will be used both by the web and the print version.
% It should *not* include the \begin{document}
%
% If you want to split the blueprint content into several files then
% the current file can be a simple sequence of \input. Otherwise It
% can start with a \section or \chapter for instance.

\chapter{Divided Power Algebras}

\section{Divided Powers}

\section{Divided Power Algebras}

\section{Sub-DP Algebras}

\section{DP-Algebra Morphisms}

\section{Id\'eaux Divis\'es}
(Do we have/need this?)

\section{DP-Algebra Quotients}

\section{The Pre-Universal Character of \texorpdfstring{$\Gamma(M)$}{Gamma(M)}}

\section{Statement of Three Problems}
Let $R = R_0 + R_+$ and $S = S_0 + S_+$ be two pre-graded algebras. From now on, we will consider the algebra $T = R \otimes S$ with the pregraduation
\begin{align*}
	T_0 &= R_0 + S_0,\\
	T_+ &= (R_0 \otimes S_+) \oplus (R_+ \otimes S_0) \oplus (R_+ \otimes S_+).
\end{align*}
The maps
\begin{align*}
	\iota_1 : R &\to T &\iota_2 : S &\to T \\
			r &\mapsto r \otimes 1 &\quad s &\mapsto 1 \otimes s
\end{align*}
are morphisms of pre-graded algebras.

\begin{lemma}
	\label{lem:roby65-1}
	\lean{on_tensorProduct_unique}
	\leanok
	%\uses{def:divided_power_algebra}
	If $R$ and $S$ are DP-algebras, then $T = R \oplus S$ admits at most one DP-algebra structure for which $\iota_1$ and $\iota_2$ are DP-morphisms.
\end{lemma}

\begin{proof}\leanok
\cite[Lemme 1]{Roby65}.
\end{proof}

\begin{definition}
	\label{def:cond_tau}
	\lean{Condτ}
	\leanok
	We say that a pair of $DP$-algebras $R$ and $S$ over $A$ satisfies the condition $\tau$ if the tensor product $T = R \otimes_A S$ admits a divided power structure for which  $\iota_1$ and $\iota_2$ are DP-morphisms.
\end{definition}

\begin{definition}
	\label{def:cond_T}
	\lean{CondT}
	\leanok
	We say that a ring $A$ verifies condition $\mathcal{T}$, and denote it by $\mathcal{T}(A)$, if every pair of split $DP$-algebras $R = R_0 + R_+$ and $S = S_0 + S_+$ over $A$ satisfies condition $\tau$.
\end{definition}

\begin{lemma}
	\label{lem:roby65-2}
	\lean{on_dpalgebra_unique}
	\leanok
	%\uses{def:divided_power_algebra}
	Let $M$ be an $A$-module. $\Gamma(M)$ admits at most one divided power structure $\{\gamma_n\}_n$ such that
	\[ \gamma_n(x) = x^{[n]} \quad\text{for all } x \in M, n \ge 0.  \]
\end{lemma}

\begin{proof}\leanok
	\cite[Lemme 2]{Roby65}.
\end{proof}

\begin{definition}
	\label{def:cond_D}
	\lean{CondD}
	\leanok
	We say that a ring $A$ verifies condition $\mathcal{D}$, and denote it by $\mathcal{D}(A)$, if for every $A$-module $M$, $\Gamma(M)$ admits a divided power structure satisfying the condition from Lemma \ref{lem:roby65-2}.
\end{definition}

% I think we should delete condQ and replace it everywhere by condQSplit.
% OTherwise, add it here.
\begin{definition}
	\label{def:cond_Q}
	\lean{CondQ}
	\leanok
	We say that a ring $A$ verifies condition $\mathcal{Q}$, and denote it by $\mathcal{Q}(A)$, if every split DP-algebra over A is the quotient of a free DP-algebra over $A$.
	
	, $\Gamma(M)$ admits a divided power structure satisfying the condition from Lemma \ref{lem:roby65-2}.
\end{definition}


In \cite{Roby65}, Roby proposes the following three problems and shows that all of them have positive answers:
\begin{enumerate}
\item Does condition $\mathcal{T}(A)$ hold for every ring $A$?
\item Does condition $\mathcal{D}(A)$ hold for every ring $A$?
\item Does condition $\mathcal{Q}(A)$ hold for every ring $A$?
\end{enumerate}

\section{Proof Ingredients}

\section{The DP-Algebra \texorpdfstring{$\Gamma(M)$}{Gamma(M)}}

\section{Tensor Products of DP-Algebras}
