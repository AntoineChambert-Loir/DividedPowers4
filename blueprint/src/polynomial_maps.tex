% In this file you should put the actual content of the blueprint.
% It will be used both by the web and the print version.
% It should *not* include the \begin{document}
%
% If you want to split the blueprint content into several files then
% the current file can be a simple sequence of \input. Otherwise It
% can start with a \section or \chapter for instance.

\chapter{Polynomial Maps}

Let $R$ be a unitary commutative ring. Unless otherwise specified, all $R$-algebras will be assumed to be commutative, associative and unitary.

\begin{definition}(\cite[\S I.2]{Roby63})
	\label{def:polynomial_map}
	\lean{PolynomialMap}
	\leanok
	Let $M$ and $N$ be two $R$-modules. A \emph{polynomial map} from $M$ to $N$ is a family of maps $f_S : S \otimes_R M \to S \otimes_R N$ for all $R$-algebras $S$, such that for any $R$-algebra morphism $u : S \to S'$, the following diagram commutes:
	\[ \begin{tikzcd}
	S \otimes_R M \arrow[r, "f_S"] \arrow[d, "\text{id}_S \otimes_R u"]
	& S \otimes_R N \arrow[d, "\text{id}_{S'} \otimes_R u"] \\
	S' \otimes_R M \arrow[r, "f_{S'}" ]
	& S' \otimes_R N
	\end{tikzcd}\]
	 
	By abuse of notation, if $z \in S \otimes_R M$, the element $f_S(z)$ of $S \otimes_R N$ will be denoted by $f(z)$, and the map $f_S$ will also be denoted by $f$.
\end{definition}

\begin{theorem}(\cite[Th\'eor\`eme I.1]{Roby63})
Let $f$ be a polynomial map from $M$ to $N$. Let $x_1, \dots, x_n$ be a finite family of elements of $M$. For each tuple $(k_1, \cdots, k_n)$ of nonnegative integers, 
there exists an element $y_{k_1, \dots, k_n}$ of $N$ such that
\begin{enumerate}
\item all but finitely many of the  $y_{k_1, \dots, k_n}$ are zero, and
\item for every $R$-algebra $S$ and every choice $s_1, \cdots s_n$ of elements of $S$, we have an equality
\[ f_S(s_1 \otimes x_1, \dots, s_n \otimes x_n) = \sum_{k_1, \dots, k_n} y_{k_1, \dots, k_n} \otimes s_1^{k_1} \cdots s_n^{k_n}. \]
\end{enumerate}

Such a family of elements $y_{k_1, \dots, k_n}$ is unique and only dependes on $x_1, \dots, x_n$.

%Conversely, if the $x_1, \dots, x_n$ are an $R$-basis of $M$, then for any choice of elements $y_{k_1, \dots, k_n}$ (almost all zero), the above formula defines a polynomial map from $M$ to $N$.
\end{theorem}

\begin{theorem}(\cite[Th\'eor\`eme I.1]{Roby63})
\label{thm:polynomial_map}
\lean{PolynomialMap.Finsupp.polynomialMap}
\lean{PolynomialMap.Finsupp.polynomialMap_toFun}
\lean{PolynomialMap.Finsupp.polynomialMap_isCompat}
\leanok
Let $x_1, \dots, x_n$ be an $R$-basis of $M$. Then for any choice of elements $y_{k_1, \dots, k_n}$, of which only finitely many are nonzero, the above formula defines a polynomial map from $M$ to $N$.	
\end{theorem}

\begin{proof}

\end{proof}

\begin{definition}
	\label{def:divided_powers_bot}
	\lean{dividedPowersBot}
	%\uses{def:divided_powers}
	\leanok
	The divided power structure on the zero ideal of a commutative ring $A$.
\end{definition}